\documentclass[spanish]{udpreport}
\usepackage[utf8]{inputenc}
\usepackage[spanish]{babel}

% Podemos establecer el logo de alguna entidad o dejar el de la UDP (defecto)
%\setlogo{EITFI}

\title{Informe Laboratorio I \\ Redes de Datos}
\author{Arturo Mantinetti \\ Manuel Tobar \\ Diego Vilches}
\email{arturo.mantinetti@mail.udp.cl \\ manuel.tobar@mail.udp.cl
	\\ diego.vilches@mail.udp.cl}
	
\profesor{Profesor \\ Jaime Álvarez}
\ayudante{Ayudante \\ Maximiliano Vega}


\date{29 de Marzo de 2016}

% Además podemos establecer la facultad y escuela
% los valores por defecto son los siguientes:
%\udpschool{Escuela de Informática y Telecomunicaciones}
%\udpfaculty{Facultad de Ingeniería}
%\udpuniversity{Universidad Diego Portales}

\begin{document}
\maketitle

\tableofcontents

\chapter{Introducción}

El reporte puede tener capítulos que lo definan.

\section{¿Por qué necesito una sección?}

Cada capítulo a su vez se divide en secciones. A diferencia de un artículo cuyo elemento superior es solo una sección, este documento puede tener capítulos para organizar más información.

\chapter{Contenido}

Acá otro capítulo.

\listoffigures

\end{document}