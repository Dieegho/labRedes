\documentclass[spanish]{udpreport}
\usepackage[utf8]{inputenc}
\usepackage[spanish]{babel}

% Podemos establecer el logo de alguna entidad o dejar el de la UDP (defecto)
%\setlogo{EITFI}

\title{Informe Laboratorio I \\ Redes de Datos}
\author{Arturo Mantinetti \\ Manuel Tobar \\ Diego Vilches \\ Tercer Amigo}
\email{arturo.mantinetti@mail.udp.cl \\ manuel.tobar@mail.udp.cl
	\\ diego.vilches@mail.udp.cl \\ tercer.amigo@mail.udp.cl}
	
\profesor{Profesor \\ Jaime Álvarez}
\ayudante{Ayudante \\ Maximiliano Vega}


\date{29 de Marzo de 2016}

% Además podemos establecer la facultad y escuela
% los valores por defecto son los siguientes:
%\udpschool{Escuela de Informática y Telecomunicaciones}
%\udpfaculty{Facultad de Ingeniería}
%\udpuniversity{Universidad Diego Portales}

\begin{document}
\maketitle

\tableofcontents

\chapter{Introducción}

Este laboratorio consta de tres actividades. La primera consiste en identificar todos los equipos que están conectados a la red, los switch de la topología, el hardware de red utilizado, tomando en cuenta su marca y modelo, el tipo de cableado ocupado, junto a sus características y por último el patch panel. La segunda actividad consiste en obtener la información de los equipos ya mencionados. Ocupando el terminal, se tiene que obtener la IP y la MAC, todo esto con el objetivo de obtener una mayor comprensión de la red. Finalmente, con todos los datos recopilados en las dos actividades anteriores, se procede a diagramar un mapa de red, en el cual, esta se debe ver representada,  especificando todos los datos recopilados durante la experiencia.


\chapter{Contenido}

\section{Topologia y Hadware}

\section{Datos de equipos}


\pagebreak
\section{Mapa de Red}

\begin{center}
	\includegraphics[scale=.37]{images/mapa_red.png} 
\end{center}

\listoffigures

\end{document}