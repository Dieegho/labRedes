\documentclass[spanish]{udpreport}
\usepackage[utf8]{inputenc}
\usepackage[spanish]{babel}

% Podemos establecer el logo de alguna entidad o dejar el de la UDP (defecto)
%\setlogo{EITFI}

\title{Informe Laboratorio III \\ Redes de Datos}
\author{Arturo Mantinetti \\ Manuel Tobar \\ Diego Vilches \\ Nicolas Henriquez}
\email{arturo.mantinetti@mail.udp.cl \\ manuel.tobar@mail.udp.cl
	\\ diego.vilches@mail.udp.cl \\ nicolas.henriquez@mail.udp.cl}
	
\profesor{Profesor \\ Jaime Álvarez}
\ayudante{Ayudante \\ Maximiliano Vega}


\date{08 de Abril de 2016}

% Además podemos establecer la facultad y escuela
% los valores por defecto son los siguientes:
%\udpschool{Escuela de Informática y Telecomunicaciones}
%\udpfaculty{Facultad de Ingeniería}
%\udpuniversity{Universidad Diego Portales}

\begin{document}
\maketitle

\tableofcontents

\chapter{Introducción}

Este laboratorio consistio en crear paquetes de datos con diferentes parámetros para luego enviarlos por la red, con el fin lo lograr comprender como se conforman estos.


\chapter{Contenido}

\section{Creación de Paquetes}

  
\section{Switch}

\subsection{Envío de un paquete de datos a FF:FF:FF:FF:FF:FF}

\subsection{Envío de un paquete de datos con MAC especifica}

\subsection{Envió de un paquete de datos con una MAC fuera de la red}

\section{HUB}

\subsection{Envío de un paquete de datos a FF:FF:FF:FF:FF:FF}

\subsection{Envío de un paquete de datos con MAC especifica}

\subsection{Envió de un paquete de datos con una MAC fuera de la red}


\chapter{Conclusión}

A partir de la creación de paquetes de datos logramos comprender.....

\end{document}
