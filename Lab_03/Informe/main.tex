\documentclass[spanish]{udpreport}
\usepackage[utf8]{inputenc}
\usepackage[spanish]{babel}

% Podemos establecer el logo de alguna entidad o dejar el de la UDP (defecto)
%\setlogo{EITFI}

\title{Informe Laboratorio III \\ Redes de Datos}
\author{Arturo Mantinetti \\ Manuel Tobar \\ Diego Vilches \\ Nicolas Henriquez}
\email{arturo.mantinetti@mail.udp.cl \\ manuel.tobar@mail.udp.cl
	\\ diego.vilches@mail.udp.cl \\ nicolas.henriquez@mail.udp.cl}
	
\profesor{Profesor \\ Jaime Álvarez}
\ayudante{Ayudante \\ Maximiliano Vega}


\date{08 de Abril de 2016}

% Además podemos establecer la facultad y escuela
% los valores por defecto son los siguientes:
%\udpschool{Escuela de Informática y Telecomunicaciones}
%\udpfaculty{Facultad de Ingeniería}
%\udpuniversity{Universidad Diego Portales}

\begin{document}
\maketitle

\tableofcontents

\chapter{Introducción}

%metanle caca cabros atte. diego
%corrijan mi caca
%recuerden que siempre se habla en tercera persona
%saludos
Este laboratorio consistio en crear paquetes de datos con diferentes parámetros para luego enviarlos por la red, con el fin lo lograr comprender como se conforman y comportan estos según sus características. 
Los paquetes varían principalmente en la MAC, lo que hace que sean recibidos por distintos equipos. Para esto se ocupa Wireshark, programa con el que se puede capturar los paquetes enviados por la red. Una vez hechos y enviados los paquetes, se repite el procedimiento, sólo que esta vez, los equipos en vez de estar conectados a un switch, lo estarán a un hub. 


\chapter{Contenido}

\section{Creación de Paquetes}
Para crear un paquete con scapy, primero se tiene que ejecutar este a través de la consola con el comando "sudo scapy". Luego se crean, según las capas  del modelo OSI, las partes del paquete (no es necesario que se creen en orden). El comando Ether() se basa en la capa 2, IP() en la capa 3,ICMP() en la capa 4 y Raw() en un payload para OSI. Luego, para juntar todo en un solo paquete, se debe apilar todo lo creado anteriormente separados por un "/". En este punto si es importante que los paquetes estén en orden según capa. Finalmente, para enviar el paquete, utilizamos el comando sendp().
  
\section{Switch}

\subsection{Envío de un paquete de datos a FF:FF:FF:FF:FF:FF}

\subsection{Envío de un paquete de datos con MAC especifica}

\subsection{Envió de un paquete de datos con una MAC fuera de la red}

\section{HUB}

\subsection{Envío de un paquete de datos a FF:FF:FF:FF:FF:FF}

\subsection{Envío de un paquete de datos con MAC especifica}

\subsection{Envió de un paquete de datos con una MAC fuera de la red}


\chapter{Conclusión}

A partir de la creación de paquetes de datos logramos comprender.....

\end{document}
