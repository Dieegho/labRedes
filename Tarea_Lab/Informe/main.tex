\documentclass[spanish]{udpreport}
\usepackage[utf8]{inputenc}
\usepackage[spanish]{babel}

% Podemos establecer el logo de alguna entidad o dejar el de la UDP (defecto)
%\setlogo{EITFI}

%\title{Informe Laboratorio III \\ Redes de Datos}
%\author{Arturo Mantinetti \\ Manuel Tobar \\ Diego Vilches \\ Nicolas Henriquez}
%\email{arturo.mantinetti@mail.udp.cl \\ manuel.tobar@mail.udp.cl
%	\\ diego.vilches@mail.udp.cl \\ nicolas.henriquez@mail.udp.cl}
%	
%\profesor{Profesor \\ Jaime Álvarez}
%\ayudante{Ayudante \\ Maximiliano Vega}


%\date{14 de Abril de 2016}

% Además podemos establecer la facultad y escuela
% los valores por defecto son los siguientes:
%\udpschool{Escuela de Informática y Telecomunicaciones}
%\udpfaculty{Facultad de Ingeniería}
%\udpuniversity{Universidad Diego Portales}

\begin{document}

\begin{itemize}
	\item Se configuran las direcciones IP de 4 equipos conectados a un Switch (Max's PC, Max's Laptop, Fabian's PC, Stolen Laptop).
	\item Router
	\begin{itemize}
		\item Se configuran las siguientes direcciones IP del router:
		\begin{itemize}
			\item 192.168.0.1
			\item 172.72.0.1
		\end{itemize}
		\item Agregamos la redes a la tabla de enrutamiento para permitir que ambas redes se comuniquen entre ellas.
		\item En cada equipo se agrega la puerta de enlace correspondiente a la red en que pertenecen.
	\end{itemize}
\end{itemize}
	

\end{document}