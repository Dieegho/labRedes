\documentclass[spanish]{udpreport}
\usepackage[utf8]{inputenc}
\usepackage[spanish]{babel}

% Podemos establecer el logo de alguna entidad o dejar el de la UDP (defecto)
%\setlogo{EITFI}

\title{Informe Laboratorio IV \\ Redes de Datos}
\author{Arturo Mantinetti \\ Manuel Tobar \\ Diego Vilches \\ Nicolas Henriquez}
\email{arturo.mantinetti@mail.udp.cl \\ manuel.tobar@mail.udp.cl
	\\ diego.vilches@mail.udp.cl \\ nicolas.henriquez@mail.udp.cl}
	
\profesor{Profesor \\ Jaime Álvarez}
\ayudante{Ayudante \\ Maximiliano Vega}


\date{XX de Mayo de 2016}

% Además podemos establecer la facultad y escuela
% los valores por defecto son los siguientes:
%\udpschool{Escuela de Informática y Telecomunicaciones}
%\udpfaculty{Facultad de Ingeniería}
%\udpuniversity{Universidad Diego Portales}

\begin{document}
\maketitle

\tableofcontents
\chapter{Actividad I}
La primera actividad consiste en analizar los problemas producidos cuando una topología de red presenta enlaces redundantes y se generan bucles. Luego se comparan los resultados al incorporar el protocolo STP(IEE 802.1d) a los equipos de la red y se analizarán las ventajas que este ofrece a la hora de enfrentarse a la redundancia de la red.
 %insertar pantallazo de la red
 %si es posible rellenar con los switch  coloquenlos donde dejé un parentesis, si no borren el parentesis
La red consiste en tres switches()  conectados entre si. Luego se envían paquetes desde el switch0 hasta el switch 2 y posteriormente desde el switch2 hasta el switch1.
%responder preguntas aquí.


\chapter{Actividad II}
Considerando la topología de la actividad I, se procede a configurar el switch1 como primario y el switch2 como secundario. Para lograr esto se deben ejecutar una serie de comandos en cada switch según como se vaya a configurar.
%poner en negrita o subtitulado o la wea que sea esto:
Para el switch primario:
 \begin{center}
    Switch>Enable
Switch#configure terminal
Enter configuration commands, one per line. End with CNTL/
Switch(config)#spanning-tree vlan 1 root primary
Switch(config)#
 \end{center}
Para el switch secundario:
 \being{center}
 Switch>Enable
Switch#configure terminal
Enter configuration commands, one per line. End with CNTL/
Switch(config)#spanning-tree vlan 1 root secondary
Switch(config)#
 \being{center}
 %responder las preguntas de la actividad 2


\chapter{Actividad III}
Para esta tercera actividad, se establecerá una prioridad para cada uno de los tres switches y se analiza el comportamiento de STP en relación a esta. Para asignar una prioridad a un switch se deben ingresar estos comandos:
\being{center}
 Switch>Enable
Switch#configure terminal
Enter configuration commands, one per line. End with CNTL/
Switch(config)#spanning-tree vlan 1 priority “NUMERO”
Switch(config)#
\being{center}


\end{document}